
The risk analysis section is an essential part of any project plan. In this section, we identify potential risks that may impact the success of the mission and develop strategies to mitigate those risks. This itemized list provides a summary of the significant risks we have identified and the steps we have taken to address them.

\begin{itemize}[leftmargin=1cm, itemindent=0.25cm, noitemsep, topsep=0pt, label=$\bullet$]
\item \textbf{Launch vehicle failure}: Can result in the loss of the CanSat. Ensure that the payload is designed to withstand launch loads and the launch vehicle is reliable. No contingency plans for this scenario, organizers should provide a different means for launch if needed.
\item \textbf{Communication failure}: Critical for mission success. Factors such as interference, distance, and line of sight can impact the quality and reliability of the communication link. Mitigate risks through system checks, tests, and system deployment procedures with a checklist. Onboard self-diagnostics and visual confirmation through an addressable LED. 
\item \textbf{Power limitations}: The payload is limited in terms of the amount of power that it can generate and store. Ensure that the payload is designed to be power-efficient and has sufficient power to complete its mission. 
\item \textbf{Environmental factors}: The payload will be subjected to a range of environmental factors such as temperature, humidity, and vibration during launch and operation. Ensure that is designed to withstand these conditions and that the mission objectives are achievable under these conditions (endurance tests in different environments, simulation of high G forces, and landing tests).
\item \textbf{Technical issues}: Can arise during the design, integration, testing, and operation of the payload. Prototype the development board to pre-assemble a fully functioning payload and catch any mistakes that were introduced in the design before building the final version. Identify potential technical issues early and have contingency plans in place to address them. 
\item \textbf{Budget constraints}: Such projects are often subject to budget constraints. Ensure that the project is feasible within the available budget and prioritize mission objectives accordingly.
\end{itemize}

\subsection{Resource estimation}
To ensure that the project is completed successfully within the given budget and timeframe, it is crucial to perform a thorough estimation of the required resources. This includes identifying the necessary materials, equipment, and team members needed to complete the project, as well as the associated costs and time requirements. 

By performing a detailed resource estimation, the project team can effectively plan and allocate resources to ensure that the project is completed on time, within budget, and to the desired quality standards. 

Below is a breakdown of the resources required for the CanSat project:
\begin{itemize}[leftmargin=1cm, itemindent=0.25cm, noitemsep, topsep=0pt, label=$\bullet$]
\item Materials and components:
\begin{itemize}[label=\ding{111}, noitemsep, topsep=1pt]
\item Identify and list out all the required materials and components, including a microcontroller, sensors, batteries, antennas, radio modules, lightweight printable plastics, and other miscellaneous parts
\item Cost of materials can vary depending on the quality and quantity of components required
\end{itemize}
\item Tools and equipment:
\begin{itemize}[label=\ding{111}, noitemsep, topsep=1pt]
\item Various tools and equipment required, such as multimeters, 3d printers, oscilloscope, soldering station, and other miscellaneous parts
\end{itemize}
\item Software:
\begin{itemize}[label=\ding{111}, noitemsep, topsep=1pt]
\item Programming microcontrollers, designing 3D concepts, and analyzing data collected by the sensors
\item Using open-source software can significantly reduce the costs associated with software for the project.
\end{itemize}
\item Manpower and build time:
\begin{itemize}[label=\ding{111}, noitemsep, topsep=1pt]
\item Online tools and weekly meetings are used to ensure accuracy
\item Estimated effort of 630 hours for the entire project.
\end{itemize}
\item Testing and validation:
\begin{itemize}[label=\ding{111}, noitemsep, topsep=1pt]
\item Essential step to ensure the project performs as expected
\item May include the use of specialized equipment such as a drone or rocket
\end{itemize}
\item Shipping, transport, and logistics:
\begin{itemize}[label=\ding{111}, noitemsep, topsep=1pt]
\item Shipping and logistics costs may need to be factored into the project’s budget
\item May include shipping materials and components and travel costs for team members to attend the competition
\end{itemize}
\item Miscellaneous costs:
\begin{itemize}[label=\ding{111}, noitemsep, topsep=1pt]
\item Other costs that may need to be considered when estimating resources for the project
\end{itemize}
\end{itemize}

In conclusion, it is evident that a successful project outcome is heavily reliant on a proper estimation of resources required to complete the project. Neglecting the factors such as materials, components, tools, equipment, software, testing and validation, shipping and logistics, and other unforeseen costs can lead to budget overruns, delays, and ultimately failure to achieve project objectives.

\subsubsection{Budget}

We conducted thorough research to ensure that our budget aligns with the ideal specifications for our CanSat. This involved finding the most efficient and lightweight components without compromising performance, as well as maximizing battery life. By carefully considering each part, we aimed to optimize our budget and ensure that we could achieve our project objectives.

However, building the CanSat is just a part of the project. Therefore, our budget covers the entire project from start to finish, which can be divided into four main parts: 
\begin{itemize}[leftmargin=1cm, itemindent=0.25cm, noitemsep, topsep=0pt, label=$\bullet$]
    \item \textbf{Hardware parts}: all the necessary mechanical and electronic/electrical parts required to build both our custom development board and final payload. The cost estimation includes the purchase of a microcontroller, sensors, batteries, antennas, radio modules, lightweight printable plastics, and other miscellaneous parts.
    \item \textbf{Publicity and Branding}: covers the creation of custom t-shirts to promote our team and the event. In addition to custom t-shirts, we will also invest in creating promotional materials such as flyers, posters, and banners that showcase our team and the mission. These materials will be distributed both digitally and in person to help increase awareness of our project and the competition. Additionally, we will allocate a portion of our budget towards branding efforts such as website design and social media management to maintain a consistent and professional image for our team. Investing in Publicity and Branding will help us build a strong reputation within the community and attract potential sponsors and supporters to our cause.
    % \item \textbf{Travel Expenses}: includes the transportation costs, with tickets for the team to travel to the Competition Venue. This includes the cost of transportation by train, as well as any necessary local travel expenses.
    \item \textbf{Emergency Funds}: a contingency budget to cover any unforeseen costs that may arise during the project.
\end{itemize}

Below is a table that outlines all the anticipated expenses associated with the CanSat mission. This includes the costs of the components used to construct the CanSat, as well as any supplementary materials, equipment, and personnel necessary to complete the mission.

The components required for the project will be purchased from various suppliers such Optimus Digital, Cle\c{s}te.ro, Mouser, TME, Adafruit, and JLCPCB. It is worth noting that the team did not receive any kits from the organizers, and as a result, the team members did all the design and construction of the CanSat.

\subsubsection{External support}
The successful execution of the CanSat mission relies on the assistance and resources given by different organizations, departments, and companies. We are fortunate to have received sponsorship or in-kind support from the following entities:
\begin{itemize}[leftmargin=1cm, itemindent=0.25cm, noitemsep, topsep=0pt, label=$\bullet$]
    % \item \textbf{CoderDojo Oradea} has provided substantial financial aid, technical support, and feedback on both hardware and software. This makes them our biggest sponsor so far, allowing us to acquire all of the most expensive components of our CanSat and facilities where we can carry out our activities;
    \item \textbf{Funda\c{t}ia Comunitar\u{a} Oradea} has generously provided facilities where we can conduct our activities;
    % \item \textbf{EduTrust} has provided both financial support and working facilities for our team;
    % \item \textbf{Elektrobit} has provided financial support, access to facilities, equipment, and expertise in Embedded systems;
    \item \textbf{Depozitul de Tricouri} has agreed to provide customized T-shirts for each team member, which will help us promote our team and the event.
\end{itemize}


\subsection{Test plan}

The success of the mission depends on the thorough testing of the system before the launch. This test plan outlines the various tests that will be conducted to ensure that the payload meets the mission objectives and functions properly under different conditions. The tests are categorized into mechanical, software, communication, power, integration, ground station, battery charging, launch, and post-mission tests. The goal of these tests is to ensure that the payload is capable of collecting and transmitting data accurately, and survive the launch and landing phases. In the annex is a roadmap for testing the mission's payload, with general ideas for each test.

The general test list provides a roadmap for ensuring that the payload meets the competition requirements and can perform its intended mission. However, to ensure the mission's success, conducting more specific tests on its electronics and structure is crucial. These tests will verify that the sensors and instruments are functioning correctly and that the payload's structure can withstand the rigors of launch, landing, and environmental conditions.

Endurance tests are an essential part of the testing process as they are designed to assess the satellite's ability to withstand the harsh environmental conditions it may encounter during a typical mission. These tests are conducted to ensure that the payload components can withstand the forces of a fall, vibration, shock, and prolonged exposure to G-forces during CanSat transport. Additionally, the endurance tests provide valuable information about the limits of the payload and its performance under different conditions. 

By conducting both general and specific tests, the team can be confident that the CanSat is fully functional and prepared for a successful mission.


\subsection{Time management}
To keep the project on schedule, we've divided it into phases, from ideation to competition. Currently, we're in the design and prototyping stages, with testing and construction to follow. Details of the testing phase will be defined later. We estimate needing 630 man-hours for the project and have already invested 330. The Gantt chart tracks our progress, helping us adjust plans to meet deadlines. For a detailed timeline, see the provided chart in the Appendix. 

\begin{itemize}[leftmargin=1cm, itemindent=0.25cm, noitemsep, topsep=0pt, label=$\bullet$]
\item Team formation: 11.10.2023 - 19.11.2023
\begin{multicols}{2}[\vspace{-0.75\baselineskip}]
\begin{itemize}[label=\ding{59}, noitemsep, topsep=2pt]
\item Think of feasible mission objectives and requirements
%\item CanSat Application Work \& Registration
\end{itemize}
\end{multicols}
\vspace*{-0.75\baselineskip}
\item Ideation: 11.10.2023 - 28.04.2024
\begin{multicols}{2}[\vspace{-0.75\baselineskip}]
\begin{itemize}[label=\ding{59}, noitemsep, topsep=2pt]
\item Documentation writing (PDR, CDR, FDR)
\item Mission refinements
\end{itemize}
\end{multicols}
\vspace*{-0.75\baselineskip}
\item Design the CanSat: 15.01.2024 - 19.05.2024
\begin{multicols}{2}[\vspace{-0.75\baselineskip}]
\begin{itemize}[label=\ding{59}, noitemsep, topsep=2pt]
\item Mechanical structure 
\item Electrical system
\item Software
\item Recovery system
\item Ground support
\end{itemize}
\end{multicols}
\vspace*{-0.75\baselineskip}
\item Prototyping: 26.02.2024 - 17.04.2024
\item Testing (prototype): 10.04.2024 - 30.04.2024
\item Build the CanSat: 15.01.2024 - 15.05.2024
\item Launch Campaign: 29.05.2024 - 02.06.2024
\end{itemize}

% Our program takes into account the potential issues that may arise at each phase and the time required to address them. However, unforeseen events may occur during the project, and as such, we have developed measures for prevention and risk mitigation, which are detailed below.

% \begin{table}[htbp]
% \centering
% \arrayrulecolor{DeepSkyBlue4}
% \begin{tabular}{>{\raggedright\arraybackslash}p{8cm}>{\raggedright\arraybackslash}p{7cm}}
% \rowcolor{DeepSkyBlue4}
% \hline
% \multicolumn{1}{c}{\textbf{\color{white!50}{Major risks}}} & \multicolumn{1}{c}{\textbf{\color{white!50}{Mitigation}}} \\
% \hline
% Delays in obtaining mandatory materials and electronic components due to the global semiconductor chip shortage, resulting in unavailability and delivery delays & Place orders for materials and electronic components well in advance, find alternative components and adjust the project timeline accordingly \\
% \rowcolor{LightCyan1!50}Overtime required for the electrical design part due to thorough testing and prototyping of the onboard circuits before integration into the final design & Create a good electrical schematic and perform exhaustive checks before beginning the physical build to catch any errors \\
% Unforeseen technical difficulties that may require additional troubleshooting time and access to technical guidance from partners & Grant some extra time for troubleshooting and have access to technical guidance from our partners \\
% \rowcolor{LightCyan1!50}Adverse weather conditions during testing & Analyze the weather status in advance and plan at least one backup day for testing \\
% CanSat damage during testing due to inappropriate launch sites and lack of safety measures & Carefully choose the launch site when testing the CanSat and ensure that all safety measures are met \\
% \rowcolor{LightCyan1!50}Recovery failure due to poor GPS transmission and weak buzzer sound & Before launching, ensure that the GPS transmission is strong and the buzzer sound is loud \\
% Team member availability: Unexpected absences or departures of team members could delay the project and lead to workload imbalances & Clear communication and expectations among team members, cross-training team members on critical tasks, and having backup plans for key roles \\
% \hline
% \end{tabular}
% \caption{Unforeseen events and risk mitigation.}
% \label{tab:risks}
% \end{table}