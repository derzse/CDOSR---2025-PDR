\subsection{Data Analysis Plan}

The mission data plan lays out all the steps we will take to gather, process, and analyze the data that the mission creates. Several Python tools are used in the plan to make sure that the data is properly collected, processed, and displayed. The data plan is made up of four parts: collecting data, processing data, visualizing data, and statistical analysis.


% \begin{itemize}[leftmargin=1cm,itemindent=0.5cm, noitemsep, topsep=0pt]
%     \item \textbf{Data acquisition}: The data acquisition component of the plan utilizes a Python library to collect and store data from the CanSat sensors in real time. The library is designed to communicate with the devices on the payload and is used to keep track of data while the CanSat is in the air.
%     \item \textbf{Data processing}: The data processing component of the plan involves a Python library that processes and cleans the raw data collected by the data acquisition software. The library is used to filter out any noise or errors in the data and organize it into a format suitable for analysis. To achieve the mission objectives, a range of statistical techniques and models will be employed to filter out noise and errors in the data. This will ensure that the data is accurate and reliable for further analysis.
%     \item \textbf{Data visualization}: The data visualization component of the plan employs Python libraries to create visual representations of the data, such as easy-to-understand graphs and charts. These visualizations can help to quickly understand the data and identify patterns or trends.
%     \item \textbf{Statistical analysis}: The statistical analysis component of the plan uses Python libraries to perform statistical analyses on the data, to identify any significant relationships or patterns in the data, such as the relationship between pressure and altitude. 
% \end{itemize}

\begin{itemize}[leftmargin=1cm,itemindent=0.5cm, noitemsep, topsep=0pt]
\item \textbf{Data acquisition}: The data acquisition component of the plan utilizes a Python library to collect and store data from the CanSat sensors in real time. The library is designed to communicate with the devices on the payload and is used to keep track of data while the CanSat is in the air.
\item \textbf{Data processing}: The data processing component of the plan involves a Python library that processes and cleans the raw data collected by the data acquisition software. The library is used to filter out any noise or errors in the data and organize it into a format suitable for analysis. To achieve the mission objectives, a range of statistical techniques and models will be employed to filter out noise and errors in the data. This will ensure that the data is accurate and reliable for further analysis.
\item \textbf{Data visualization}: The data visualization component of the plan employs Python libraries to create visual representations of the data, such as easy-to-understand graphs and charts. These visualizations can help to quickly understand the data and identify patterns or trends.
\item \textbf{Statistical analysis}: The statistical analysis component of the plan uses Python libraries to perform statistical analyses on the data, to identify any significant relationships or patterns in the data, such as the relationship between pressure and altitude.
\item \textbf{Biological analysis}: The biological analysis component of the plan involves collecting bacteria from the sample collection system exposed during the CanSat mission and culturing them on petri dishes. The gauzes will be carefully transferred to a sterile environment, where samples will be placed on nutrient agar plates and incubated under controlled conditions. After incubation, bacterial colonies will be observed, and their growth patterns analyzed. This analysis will help determine the presence and diversity of microorganisms collected during the mission.
\end{itemize}

% In addition to the software and tools used, the data analysis plan also outlines potential challenges and limitations that may arise during the analysis process. One such challenge is the need to account for external factors, such as atmospheric conditions, that may affect the accuracy of the data.

% Another critical aspect of the data analysis plan is the documentation of the process to ensure transparency and reproducibility. The plan includes detailed descriptions of each step of the analysis process, including the software and tools used, any models or statistical techniques applied, and any assumptions or limitations of the analysis.

% The use of statistical techniques and models, as well as powerful Python libraries for data manipulation and visualization, will enable the mission team to draw meaningful insights from the data and make informed decisions.

