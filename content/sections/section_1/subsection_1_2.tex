\subsection{Team organization and roles}

\subsubsection{Team name}
The team name {\textbf{CDOSR}} stands for Coderdojo Oradea Space Robotics, a group focused on space and robotics technologies, originating from CoderDojo Oradea. CDOSR is also a long-time participant in the CanSat competition, showcasing its expertise and passion for space-related projects.

CoderDojo Oradea is a community programming club in Oradea, Romania, part of the global CoderDojo network that offers free coding sessions for young people aged 7-17. Its goal is to inspire young people to learn programming and explore technology in a fun and supportive environment.

CDOSR represents a branch within CoderDojo Oradea, concentrating on space robotics. It brings together young enthusiasts passionate about space exploration and robotics, providing a platform for learning, collaboration, and innovation in space-related challenges.


\subsubsection{Team composition}

The CoderDojo Oradea Space Robotics team, comprising a leader and five members aged 16-17 of various high schools in Oradea, emphasizes open communication for effective collaboration. To foster a collaborative learning environment, each team member is assigned a primary role and two secondary roles. This structure ensures a balanced distribution of responsibilities while encouraging skill development across diverse areas. Below is the team role assignment matrix, which outlines the roles and responsibilities of each member:


\begin{table}[htb]
    \centering
    \arrayrulecolor{CDOSRPrimary}
        \begin{tabular}{>{\centering\arraybackslash}l|*{6}{c}}
        \hline
        \rowcolor{CDOSRPrimary}
        \textbf{\color{white!50}\footnotesize{Role / Person}} & \textbf{\color{white!50}\footnotesize{Alin}} & \textbf{\color{white!50}\footnotesize{David}} & \textbf{\color{white!50}\footnotesize{Mark}} & \textbf{\color{white!50}\footnotesize{\c{S}tefan}} & \textbf{\color{white!50}\footnotesize{Antonio}} & \textbf{\color{white!50}\footnotesize{Miruna}} \\
        \rowcolor{CDOSRPrimary}
        & \textbf{\color{white!50}\footnotesize{Lup\u{a}u}} & \textbf{\color{white!50}\footnotesize{Chereche\c{s}}} & \textbf{\color{white!50}\footnotesize{Trefi}} & \textbf{\color{white!50}\footnotesize{Mierea}} & \textbf{\color{white!50}\footnotesize{Laza}} & \textbf{\color{white!50}\footnotesize{Ursu}} \\
        \hline
        \footnotesize{Hardware Design} & & \color{CDOSRWarning!80}\faCheckSquare & & \color{CDOSRWarning!80}\faCheckSquare & \color{CDOSRPrimary}\faCheckSquare & \\
        \rowcolor{CDOSRSecondary!50}
        \footnotesize{Software Design} \& Programming & \color{CDOSRPrimary}\faCheckSquare & \color{CDOSRWarning!80}\faCheckSquare & & & \color{CDOSRWarning!80}\faCheckSquare & \\
        \footnotesize{Data Analysis} \& Interpretation & \color{CDOSRWarning!80}\faCheckSquare & & & \color{CDOSRWarning!80}\faCheckSquare & & \color{CDOSRPrimary}\faCheckSquare \\
        \rowcolor{CDOSRSecondary!50}
        \footnotesize{3D Modeling} & & & \color{CDOSRPrimary}\faCheckSquare & & \color{CDOSRWarning!80}\faCheckSquare & \color{CDOSRWarning!80}\faCheckSquare \\
        \footnotesize{Base Operator, Radio} & \color{CDOSRWarning!80}\faCheckSquare & \color{CDOSRPrimary}\faCheckSquare & \color{CDOSRWarning!80}\faCheckSquare & & & \\
        \rowcolor{CDOSRSecondary!50}
        \footnotesize{Social Networking} & & & \color{CDOSRWarning!80}\faCheckSquare & \color{CDOSRPrimary}\faCheckSquare & & \color{CDOSRWarning!80}\faCheckSquare \\
        \hline
        \end{tabular}
        \caption{\small{Team Role Assignment Matrix}}
    \end{table}

% \vspace{0.5cm}
\begin{itemize}[leftmargin=*]
    %    \item[] \textbf{Daniel Erzse (teacher)}
    \begin{itemize}[label=\ding{109}]
        \item[\faCogs] \textbf{Previous Experience:} 
        \begin{itemize}[label=\ding{59}]
            \item Taught Advanced Mathematics, Statistics, and Astronomy as a university lecturer for over 17 years.
            \item Coordinates and mentors the local CoderDojo in Oradea for the last four years.
            \item Led CoderDojo teams in Astro Pi and CanSat competitions, as well as in the Exo-RO Rover Challenge.
        \end{itemize}
        \item[\faGraduationCap] \textbf{Background and Interests:} 
        \begin{itemize}[label=\ding{59}]
            \item STEM teacher with a passion for hands-on projects as the best way to teach and learn.
        \end{itemize}
        \item[\faMicroscope] \textbf{Field of Work (Role):} Team leader and software advisor. Coaching and coordinating the team.
        \item[\faLaptopCode] \textbf{Expected Workload:} 10 hours/week (3 hours/week at CoderDojo, 7 at home)
    \end{itemize}
    \vspace{0.2 cm}

    \item[] \textbf{Alin Lup\u{a}u}
    \begin{itemize}[label=\ding{109}]
        \item[\faCogs] \textbf{Previous Experience:} Participation in CoderDojo Coolest Project International (Dublin, 2018)
.        % \begin{itemize}[label=\textbullet]
        %     \item Participated in the CanSat competition as a team member of CDOSR in 2022.
        % \end{itemize}
        \item[\faGraduationCap] \textbf{Background and Interests:} 
        \begin{itemize}[label=\textbullet]
            \item 9th-grade student at “Emanuil Gojdu” National College, Oradea.
            \item Enthusiastic about delving into the realms of physics, computer science, rocketry, and anything associated with these fields.
            \item Thinks that engaging in competitions is an excellent means of personal enhancement.
            \item Considers the significance of continual improvement and aspiring to evolve into a better individual each day.
        \end{itemize}
        \item[\faEdit] \textbf{Contribution:} In charge of mechanical and rocketry designs and contributing to programming tasks. 
        \item[\faMicroscope] \textbf{Field of Work (Role):} Software design and high-level programming, Mechanical design.
    \end{itemize}
    \vspace{0.2 cm}
    
    \item[] \textbf{David Chereche\c{s}}
    \begin{itemize}[label=\ding{109}]
        \item[\faCogs] \textbf{Previous Experience:} Participation and awards in First Lego League (2019) and RC Robotics Championship (2023)
        % \begin{itemize}[label=\ding{59}]
        %     \myitemtwo Has two years of membership in the FTC team, Modus Vivendi, at "Mihai Eminescu" National College.
        % \end{itemize}
        \item[\faGraduationCap] \textbf{Background and Interests:} 
        \begin{itemize}[label=\textbullet]
            \item 9th-grade student at ”Mihai Eminescu” National College, Oradea.
            \item Participated in the “Meridian Zero” astronomy club, being interested to learn more about the universe.
            \item Eager to learn more about programming and improve his skills.
            \item Believes it’s important to improve every day and become a better person.
        \end{itemize}
        \item[\faEdit] \textbf{Contribution:} Responsible for programming the research module, analyzing the data received from the payload, and managing the team’s public image.
        \item[\faMicroscope] \textbf{Field of Work (Role):} Software design, data analysis, outreach planning.
    \end{itemize}
    \vspace{0.2 cm}
    
    \item[] \textbf{Mark Trefi}
    \begin{itemize}[label=\ding{109}]
        % \item[\faCogs] \textbf{Previous Experience:} 
        % \begin{itemize}[label=\ding{59}]
        %     \item Former member of the winning team at ExoRo 2021.
        %     \item Team member in the CoderDojo Oradea and Modus Vivendi joint team, participating in the Qube2Space competition in 2022.
        % \end{itemize}
        \item[\faGraduationCap] \textbf{Background and Interests:} 
        \begin{itemize}[label=\textbullet]
            \item 10th-grade student at ”Traian Vuia” Technical College, Oradea.
            \item Open to learn about foreign topics of any kind.
            \item Has a “Jack of all trades, master of none” ideology and believes that learning about a broad range of things is essential.
        \end{itemize}
        \item[\faEdit] \textbf{Contribution:} Responsible for the mechanical design, including wiring, layout and schematics, outreach planning, and public relations management, like graphic design and website management. He will also contribute by creating digital simulations and models of the rocket.
        \item[\faMicroscope] \textbf{Field of Work (Role):} Mechanical design, outreach projects, and computer modeling.
    \end{itemize}
    \item[] \textbf{Stefan Mierea}
    \begin{itemize}[label=\ding{109}]
        % \item[\faCogs] \textbf{Previous Experience:} 
        % \begin{itemize}[label=\textbullet]
        %     \item Participated in Coolest Projects Romania and Coolest Projects Dublin in the Hardware section of the competition in 2017 and 2018.
        %     \item Participated in the CanSat and Exo-Ro competitions as a team member of CDOSR in 2021.
        % \end{itemize}
        \item[\faGraduationCap] \textbf{Background and Interests:} 
        \begin{itemize}[label=\textbullet]
            \item 10th-grade student at “Mihai Eminescu” National College, Oradea.
            % \item Motivated to learn more about the electronic part.
        \end{itemize}
        \item[\faEdit] \textbf{Contribution:} Circuit/electrical design. Responsible for the schematics and sensor footprint design for the payload (CanSat).
        \item[\faMicroscope] \textbf{Field of Work (Role):} Electronics design \& integration.
    \end{itemize}
    \vspace{0.2 cm}
    
    \item[] \textbf{Antonio Laza}
    \begin{itemize}[label=\ding{109}]
        \item[\faCogs] \textbf{Previous Experience:} 
            \begin{itemize}[label=\textbullet]
                \myitemtwo Participated in multiple projects for ESA's AstroPi program.
                \myitemtwo  Qube2Space (2nd place, 2024),
                \myitemtwo European Space Design Competition (EUSDC 2025).
            \end{itemize}
        \item[\faGraduationCap] \textbf{Background and Interests:} 
            \begin{itemize}[label=\textbullet]
                \myitemtwo 10th-grade student at "Emanuil Gojdu" National College, Oradea,
                \myitemtwo Has a great interest in Physics, Astronomy, and Programming.
            \end{itemize}
        \item[\faEdit] \textbf{Contribution:}
            \begin{itemize}[label=\textbullet]
            \myitemtwo Leads software development and programming for the CanSat system,
            \myitemtwo Contributes to hardware design and electrical components implementation,
            \myitemtwo Creates 3D models for mechanical components and structural elements.
            \end{itemize}
        \item[\faMicroscope] \textbf{Field of Work (Role):} Software design and programming, hardware design, 3D modeling.
    \end{itemize}
    \vspace{0.2 cm}
    
    \item[] \textbf{Miruna Ursu}
    \begin{itemize}[label=\ding{109}]
        % \item[\faCogs] \textbf{Previous Experience:} 
        % \begin{itemize}[label=\ding{59}]
        %     \item Assisted other teams at CDOSR to gain experience and knowledge.
        % \end{itemize}
        \item[\faGraduationCap] \textbf{Background and Interests:} 
        \begin{itemize}[label=\textbullet]
            \item 10th-grade student at ”Emanuil Gojdu” National College, Oradea.
            % \item Considers competitions a great opportunity to learn new things.
            % \item Interested in developing skills related to programming and technology.
            % \item Believes that every moment is an opportunity to learn and grow, and she is eager to seize each one to develop her skills and knowledge further.
            % \item Member of CoderDojo since 2021.
        \end{itemize}
        \item[\faEdit] \textbf{Contribution:} Responsible for the flawless functioning of the recovery system, monitoring and controlling the rocket and ground station. Assist with data analysis and interpretation, using software tools to analyze the data transmitted from the payload.
        \item[\faMicroscope] \textbf{Field of Work (Role):} Recovery system, ground station, data analysis and interpretation.
    \end{itemize}
    \vspace{0.2 cm}
    
\end{itemize}

\subsubsection{Team's activity} 

The team has developed a six-point action plan to ensure the successful design, development, and testing of the CanSat. This structured approach emphasizes collaboration, planning, and continuous improvement to meet competition requirements and achieve optimal performance. The plan includes the following key steps:
\begin{enumerate}[topsep=3pt]
    \item Effective Collaboration 
        \begin{itemize}[leftmargin=0.25cm,itemindent=0.5cm, noitemsep, topsep=1pt, label=\textbullet]
            \myitemtwo Leverage the unique skills and expertise of each team member by assigning clear roles and responsibilities aligned with their strengths and interests.
            \myitemtwo Ensure every member understands their tasks and contributions to the project.
            \myitemtwo Create a team environment that encourages open communication, values diverse perspectives, and promotes constructive feedback.
        \end{itemize}
    \item Establish a clear project plan and timeline 
        \begin{itemize}[leftmargin=0.25cm,itemindent=0.5cm, noitemsep, topsep=1pt, label=\textbullet]
            \myitemtwo Develop a detailed roadmap outlining specific milestones, deadlines, and deliverables for each phase of the project (design, development, and testing).
            \myitemtwo Allocate sufficient time for iterative improvements and unexpected challenges.
        \end{itemize}
    \item Maintain regular communication and progress updates
        \begin{itemize}[leftmargin=0.25cm,itemindent=0.5cm, noitemsep, topsep=1pt, label=\textbullet]
            \myitemtwo Schedule regular team meetings to discuss progress and challenges.
            \myitemtwo Provide updates on individual tasks and responsibilities to ensure that the team is working efficiently and effectively.
            \myitemtwo Encourage open communication and idea sharing to facilitate collaboration and problem solving.
        \end{itemize}
    \item Utilize diverse resources for continuous learning
        \begin{itemize}[leftmargin=0.25cm,itemindent=0.5cm, noitemsep, topsep=1pt, label=\textbullet]
            \myitemtwo Actively seek and employ online tools, tutorials, and resources to support the design and development process.
            \myitemtwo Stay informed about industry trends, technological advancements, and best practices to enhance the team’s knowledge and innovation.
        \end{itemize}
    \item Implement rigorous testing and evaluation
        \begin{itemize}[leftmargin=0.25cm,itemindent=0.5cm, noitemsep, topsep=1pt, label=\textbullet]
            \myitemtwo Develop a comprehensive testing plan to ensure the CanSat meets competition requirements and performs as intended.
            \myitemtwo Regularly evaluate the design to identify areas for improvement and address potential issues.
            \myitemtwo Iterate and refine the design based on testing results and feedback.
        \end{itemize}
    \item Seek mentorship and expert guidance
        \begin{itemize}[leftmargin=0.25cm,itemindent=0.5cm, noitemsep, topsep=1pt, label=\textbullet]
            \myitemtwo Regularly seek feedback and input from mentors and advisors to enhance the design and development process.
            \myitemtwo Utilize feedback and input to improve the design and development process and enhance the team's knowledge and expertise.
        \end{itemize}
\end{enumerate}
