\documentclass[11pt]{article}
\usepackage{CDOSR_CanSat, lipsum}
\cansatstyle

\renewcommand{\labelenumii}{\theenumii}
\renewcommand{\theenumii}{\theenumi.\arabic{enumii}.}

\title{Evaluation Process of CanSat PDRs}
\author{Team: ROSPIN}
\date{\today}

\begin{document}

\cansattitle{}{}

\tableofcontents
\pagestyle{plain}

\newpage


The Preliminary Design Review (PDR) is a critical stage in the development of a CanSat mission, where the team presents their detailed design and plan to a panel of experts for evaluation. 

The PDR evaluates the feasibility and effectiveness of the mission concept, and ensures that the team has a comprehensive understanding of the technical and operational aspects of their CanSat. 

To evaluate the PDRs, a set of criteria is used, covering various aspects of the CanSat design and mission plan. Each criterion is assigned a score based on the quality and completeness of the information provided. 

The evaluation process is objective and impartial, and aims to identify the strengths and weaknesses of each proposal, as well as provide feedback and recommendations to improve the overall design and plan. The evaluation process is crucial for ensuring the success of the mission, as it helps the team to refine their design and plan, and address any issues or challenges that may arise during the mission. 

Ultimately, the evaluation process enables the team to present a well-designed and comprehensive CanSat mission that can meet the objectives and expectations of the scientific community.

\section{Importance of Evaluation}
The evaluation process is crucial in ensuring the success of a CanSat mission for several reasons. Firstly, having an objective and impartial evaluation process helps to ensure that the mission is feasible and effective. The evaluation process is conducted by a panel of experts who have experience and knowledge in the field of CanSat missions. They review the proposal and provide an impartial evaluation, which helps to identify any potential issues or challenges that may arise during the mission.

Secondly, the evaluation process helps to identify the strengths and weaknesses of each proposal. This helps the team to refine their design and plan by addressing any issues or challenges that were identified during the evaluation. The feedback and recommendations provided by the evaluation panel enable the team to make improvements to their proposal, which can increase the likelihood of a successful launch.

Finally, the evaluation process enables the team to refine their design and plan, which can improve the overall quality of the mission. The evaluation panel provides feedback and recommendations that can help the team to enhance their proposal in areas such as mission objectives, mechanical and structural design, electrical and software design, recovery system, ground support, and time management. 
%By refining their design and plan, the team can increase the chances of a %successful launch and ensure that the mission meets the objectives and %expectations of the scientific community.

In conclusion, the evaluation process is critical in ensuring the success of a CanSat mission. It helps to ensure that the mission is feasible and effective, identify the strengths and weaknesses of each proposal, and refine the design and plan to improve the overall quality of the mission. By conducting a thorough and objective evaluation process, the team can increase the likelihood of a successful launch and meet the objectives and expectations of the scientific community.

\section{Overview of Criteria}
The evaluation criteria used to assess the CanSat PDRs cover various aspects of the mission design and plan. Here's an overview of each criterion and its importance in evaluating the CanSat proposal:

\begin{enumerate}
    \item {\textbf{Mission and team} (10 \% of the final score)}
    \begin{enumerate}
        \item {\textbf{Purpose of mission:}} This criterion evaluates the clarity and feasibility of the mission's objective and its potential impact on the scientific community. It is important to ensure that the mission's purpose is well-defined and significant to the scientific community, as this will help to ensure the mission's success.

        \item {\textbf{Team organization:}} This criterion evaluates the overall organization of the team, including the roles and responsibilities of each member, communication, and coordination. It is important to have a well-organized team with a clear division of labor and effective communication to ensure that the mission is conducted smoothly and efficiently.

        \item {\textbf{Mission objectives:}} This criterion evaluates the completeness and relevance of the mission objectives to the purpose of the mission. It is important to have comprehensive and relevant mission objectives that are aligned with the purpose of the mission to ensure that the mission meets its goals.
    \end{enumerate}
    
    \item {\textbf{CanSat description} (33 \% of the final score)}
    \begin{enumerate}
        \item {\textbf{Mission overview:}} This criterion evaluates the overall concept and strategy of the mission. It is important to have a concise and clear mission overview that outlines the key components of the mission and how they will be executed to ensure that the mission is well-planned and organized.

        \item {\textbf{Mechanical and structural design:}} This criterion evaluates the mechanical and structural design of the CanSat. It is important to have a well-designed CanSat with robust and reliable materials to ensure that the CanSat can withstand the harsh conditions of the mission.

        \item {\textbf{Electrical design:}} This criterion evaluates the electrical design of the CanSat, including power management, sensors, and communication. It is important to have a well-designed electrical system with efficient power management and reliable communication to ensure that the CanSat can function effectively during the mission.

        \item {\textbf{Software design:}} This criterion evaluates the software design of the CanSat, including programming and algorithms. It is important to have a well-designed software system with accurate and efficient programming and algorithms to ensure that the CanSat can perform its tasks effectively.

        \item {\textbf{Recovery system:}} This criterion evaluates the recovery system of the CanSat, including the parachute, landing gear, and telemetry. It is important to have a well-designed recovery system with reliable and safe landing and accurate telemetry to ensure that the CanSat can be recovered after the mission.

        \item {\textbf{Ground support:}} This criterion evaluates the ground support equipment and infrastructure needed for the mission. It is important to have efficient and adequate ground support to ensure that the mission can be conducted safely and effectively.
    \end{enumerate}

    \item {\textbf{Project planning} (34 \% of the final score)}
    \begin{enumerate}
        \item {\textbf{Time schedule:}} This criterion evaluates the feasibility and accuracy of the time schedule for the mission, including the launch and recovery dates. It is important to have a realistic and detailed schedule to ensure that the mission is conducted efficiently and within the allotted time frame.
        
        \item {\textbf{Risks analysis:}} This criterion evaluates the risks analysis of the mission, including potential hazards and safety measures. It is important to have a comprehensive and realistic risks analysis with adequate safety measures to ensure that the mission can be conducted safely.
        
        \item {\textbf{Budget:}} This criterion evaluates the budget allocation for the mission, including materials, equipment, and travel expenses. It is important to have a well-planned and detailed budget with reasonable expenses to ensure that the mission can be conducted within the allocated budget.
        
        \item {\textbf{External support:}} This criterion evaluates the external support received from sponsors or partners, including financial or technical assistance. It is important to have significant and relevant external support to ensure that the mission can be conducted effectively and efficiently.
        
        \item {\textbf{Test plan:}} This criterion evaluates the test plan for the CanSat, including the testing of individual components and the integrated system. It is important to have a comprehensive test plan to ensure that the CanSat can function effectively during the mission.
        
        \item {\textbf{Time management:}} This criterion evaluates the time management plan for the mission. It is important to have an effective time management plan to ensure that the mission is conducted efficiently and within the allotted time frame.
    \end{enumerate}

    \item {\textbf{Data analysis:} (17 \% of the final score)} 
    \begin{itemize}
        \item[] This criterion evaluates the data analysis plan for the mission, including the methods and tools for data collection and analysis. It is important to have a comprehensive data analysis plan to ensure that the data collected during the mission is analyzed effectively and accurately.
    \end{itemize}

    \item {\textbf{Outreach:} (6 \% of the final score)} 
     \begin{itemize}
        \item[] This criterion evaluates the outreach plan for the mission, including the dissemination of results and engagement with the scientific community. It is important to have an effective outreach plan to ensure that the mission's results are communicated to the scientific community and the general public.
    \end{itemize}    
\end{enumerate} 

Overall, each criterion plays an important role in evaluating the CanSat proposal and contributes to the overall mission design and plan. By assessing each criterion, the evaluation panel can identify areas of strength and weakness and provide feedback and recommendations to help refine the design and plan for a successful launch.

\section{Scoring System}
The scoring system used to evaluate the CanSat PDRs assigns points to each criterion based on the quality and completeness of the information provided. The points assigned to each criterion can vary and are based on the quality and completeness of the information provided in the proposal.

The total score is then tallied to determine the overall quality of the proposal. The scoring system enables the evaluators to rank the proposals based on their scores and determine which proposals are most promising.
The scoring system enables the evaluators to objectively and comprehensively assess each proposal by providing a structured and consistent approach to evaluating each criterion. The scoring system ensures that each proposal is evaluated using the same criteria and that the evaluators are consistent in their evaluation. This reduces the potential for bias or subjective evaluation.

Furthermore, the scoring system enables the evaluators to assess each proposal comprehensively by assigning points to each criterion. This ensures that each aspect of the mission design and plan is evaluated and considered. The scoring system provides a clear and transparent method of evaluating each proposal, enabling the team to understand their strengths and weaknesses and make improvements where necessary.

The scoring system also allows for easy comparison of different proposals, as the total score provides a quantitative measure of the proposal's quality. This helps the evaluation panel to rank the proposals based on their scores and determine which proposals are most promising.

In conclusion, the scoring system used to evaluate the CanSat PDRs provides a structured and consistent approach to evaluating each proposal. It enables the evaluators to objectively and comprehensively assess each proposal and compare different proposals based on their scores. The scoring system is an important tool in ensuring a fair and transparent evaluation process and helps to identify the strongest proposals for a successful launch.

\section{Feedback and Recommendations}
The evaluation process provides feedback and recommendations to the team through the evaluation panel's assessment of each criterion. After reviewing the CanSat PDR, the panel provides a score for each criterion and provides feedback on areas of strength and weakness. The feedback is usually in the form of written comments, but may also include verbal feedback during the evaluation process.

The feedback helps the team to identify areas for improvement and refine their design and plan. For example, if the panel identifies weaknesses in the mechanical and structural design of the CanSat, the team may focus on improving the design to ensure that it can withstand the harsh conditions of the mission. If the panel identifies weaknesses in the time management plan, the team may revise the plan to ensure that the mission is conducted efficiently and within the allotted time frame.

The feedback and recommendations provided by the evaluation panel can take many forms, depending on the specific competition or program. Some examples of the types of feedback and recommendations that may be provided include:

\begin{itemize}
    \item {\textbf{Technical feedback:}} The panel may provide technical feedback on the design and functionality of the CanSat, including suggestions for improvements or changes to the design.

    \item {\textbf{Operational feedback:}} The panel may provide feedback on the operational aspects of the mission, including the time management plan, recovery system, and ground support.

    \item {\textbf{Safety feedback:}} The panel may provide feedback on the safety measures included in the CanSat design and recommend additional safety measures to ensure the safety of the mission.

    \item {\textbf{Budget feedback:}} The panel may provide feedback on the budget allocation for the mission and recommend ways to reduce costs or allocate funds more effectively.

    \item {\textbf{Outreach feedback:}} The panel may provide feedback on the outreach plan for the mission and recommend ways to engage with the scientific community and the general public.
\end{itemize}

By incorporating the feedback and recommendations provided by the evaluation panel, the team can refine their design and plan and enhance the overall quality of the mission. The feedback helps the team to identify areas of improvement and make changes that can increase the likelihood of a successful launch. Ultimately, the feedback and recommendations provided through the evaluation process enable the team to present a well-designed and comprehensive CanSat mission that can meet the objectives and expectations of the scientific community.



\end{document}
